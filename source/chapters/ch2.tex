% !Mode:: "TeX:UTF-8"

\chapter{USB转接器的整体方案设计}
\section{USB简介}
随着电子科技的发展与应用,各种计算机外围接口不断推陈出新,USB接口已经成为现今个人计算机上最重要的接口之一,各种电子消费产品也逐渐配置这种接口。USB接口是速度比较高的串行接口,具有较广阔的发展前景和应用潜力。USB适用于低档外设与主机之间的高速数据传输。从USB问世至今,USB在不断的自我完善,并走向成熟。从普通计算机用户、计算机工程师、到硬件芯片生产厂商,都已经完全认可了USB。

与其它通信接口比较,USB接口的最大特点是易于使用,这也是USB的主要设计目标。作为一种高速总线接口,USB适用于多种设备,如数码相机、MP3播放机、高速数据采集设备等。易于使用还表现在USB接口支持热插拔,并且所有的配置过程都由系统自动完成,无需用户干预。USB接口支持1.5Mb/s(低速)和12Mb/s(全速)的数据传输速率,扣除用于总线状态、控制和错误监测等的数据传输,USB的最大理论传输速率仍达1.2Mb/s或9.6Mb/s,远高于一般的串行总线接口。USB接口芯片价格低廉,这也大大促进USB设备的开发与应用。 
在进行一个USB设备开发之前,首先要根据具体使用要求选择合适的USB控制器。目前,市场上供应的USB控制器主要有两种:带USB接口的单片机(MCU)或纯粹的USB接口芯片。带USB接口的单片机从应用上又可以分成两类,一类是从底层设计专用于USB控制的单片机,另一类是增加了USB接口的普通单片机,如Cypress公司的EZ-USB(基于8051),选择这类USB控制器的最大好处在于开发者对系统结构和指令集非常熟悉,开发工具简单,但对于简单或低成本系统。其价格因素也是在实际选择过程中需要考虑的因素。纯粹的USB接口芯片仅处理USB通信,必须有一个外部微处理器来进行协议处理和数据交换。这类典型产品有Philips公司的PDIUSBD12(并行接口),NS公司USBN9603/9604(并行接口),NetChip公司的NET2888等。USB接口芯片的主要特点是价格便宜、接口方便、可靠性高,尤其适合于产品的改型设计(硬件上仅需对并行总线和中断进行改动,软件则需要增加微处理器的USB中断处理和数据交换程序,PC机的USB接口通信程序,无需对原有产品系统结构作很大的改动)\citeup{USBcomplete}。
\section{USB体系及协议}
USB以USB主机为核心,以外围的USB设备为功能,组成了USB系统模型。主机是USB的核心,每次USB数据通信都必须是由USB主机来发起的,主机管理着USB设备。USB物理上是一个含有两条电源线(VCC,GND)和两条以差分方式产生信号的线(D+,D-),传输率可达12Mbps的串行接口,一个PC主机可以连接多达127个外围设备。USB协议是以令牌包为主的通信协议,12Mbps的总线带宽被分割成1ms的帧,所有任务以时分传输(TDM)来分享总线。

 
\subsection{USB体系概述}
\subsubsection{USB系统描述}
一个USB系统主要被定义为三个部分:
\begin{enumerate}
    \item USB的互连
    \item USB的设备
    \item USB的主机
\end{enumerate}

USB的互连是指USB设备与主机之间进行连接和通信的操作,主要包括以下几方面: 

总线的拓扑结构:USB设备与主机之间的各种连接方式; 

\begin{enumerate}
\item 内部层次关系
根据性能叠置,USB的任务被分配到系统的每一个层次
\item 数据流模式
描述了数据在系统中通过USB从产生方到使用方的流动方式
\item USB的调度
USB提供了一个共享的连接。对可以使用的连接进行了调度以支持同步数据传输,并且避免的优先级判别的开销
\end{enumerate}

USB连接了USB设备和USB主机,USB的物理连接是有层次性的星型结构。每个网络集线器是在星型的中心,每条线段是点点连接。从主机到集线器或其功能部件,或从集线器到集线器或其功能部件,从图\ref{USBjinzita}中可看出USB系拓扑结构。

其中,USB集线器Hub是一组设备的连接点,主机中有一个被嵌入的Hub叫根Hub(root Hub)。主机端通常是指PC主机或是另外再附加USB 端口的扩充卡,主机通过根Hub提供若干个连接点。集线器除了扩增系统的连接点外,还负责中继(repeat)上游或下游的信号以及控制各个下游端口的电源管理。 

当PC上电时,所有USB设备与Hub都默认地址为0,PC机启动程序向USB查询,地址1分配给发现的第一个设备,地址2分配给第二个设备或Hub,如此重复寻找并分配地址,直到所有设备赋完地址,并加载相应的的驱动程序。 

当设备突然被拔移后,PC机通过D+或D-的电压变化检测到设备被移除掉后,将其地址收回,并列入可使用的地址名单中。 

在任何USB系统中,只有一个主机。USB和主机系统的接口称作主机控制器,主机控制器可由硬件、固件和软件综合实现。根集线器是由主机系统整合的,用以提供更多的连接点。 

\pic[h]{总线的拓扑结构}{width=4in}{USBjinzita}

USB的设备如下所示: 
\begin{enumerate}
\item 网络集线器,向USB提供了更多的连接点

\item 功能器件:为系统提供具体功能,如数字的游戏杆或扬声器
\end{enumerate}

USB设备提供的USB标准接口的主要依据:     
\begin{enumerate}
\item 对USB协议的运用

\item 对标准USB操作的反馈,如设置和复位

\item 标准性能的描述性信息 
\end{enumerate}

\subsubsection{USB连接头机器供电方式}
为了避免连接错误,USB定义了两种不同规格的星形USB连接头:序列A与B连接头,其中序列A接头用来连接下游的设备。每个连接头内拥有4个针脚,其中两个是用来传递差分数据的,另两个则用于USB设备的电源供给。 

USB的供电方式有两种: 
\begin{enumerate}
\item 总线供电集线器 
电源由上游连接端口供应,最多只能从上游端消耗500mA。一个4个连接端口的集线器,每个下游端口最多消耗为100mA,外围电路消耗100mA。 
\item 自我供电集线器 
集线器本身有电源,可以提供给本身的控制器以及下游端口至少500mA的电流,集线器最多可从上游端消耗100mA。 
\end{enumerate}
\subsubsection{USB系统软硬件组成}
USB系统的软硬件资源可以分为3个层次:功能层、设备层和接口层。 

功能层提供USB设备所需的特定的功能,主机端的这个功能由用户软件和设备类驱动程序提供,而设备就由功能单元来实现。 

设备层主要提供USB基本的协议栈,执行通用的USB的各种操作和请求命令。从逻辑上讲,就是USB系统软件与USB逻辑设备之间的数据交换。 

接口层涉及的是具体的物理层,其主要实现物理信号和数据包的交互,即在主机端的USB主控制器和设备的USB总线接口之间传输实际的数据流。 

无论在软件还是硬件层次上,USB主机都处于USB系统的核心。主机系统不仅包含了用于和USB外设进行通信的USB主机控制器及用于连接的USB接口(SIE),更重要的是,主机系统是USB系统软件和USB客户软件的载体。USB主机软件系统可以分为三个部分:      
\begin{enumerate}
\item 客户软件部分(CSW)
在逻辑上和外设功能部件部分进行资料的交换

\item USB系统软件部分(即HCDI)
在逻辑和实际中作为HCD 和USBD之间的接口

\item USB主机控制器软件部分(即HCD和USBD)
用于对外设和主机的所有USB有关部分的控制和管理,包括外设的SIE部分、USB资料发送接收器(Transreceiver)部分及外设的协议层等\citeup{USBcomplete}。
\end{enumerate}


\section{RT-Thread简介}



\section{STM32单片机简介}















