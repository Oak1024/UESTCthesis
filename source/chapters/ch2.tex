% !Mode:: "TeX:UTF-8"

\chapter{USB转接器的整体方案设计}

\section{USB转接器的设计内容}
本课题制作了一个基于RTOS的智能USB转接器。其结构图如下:

\pic[h]{转接器示意图}{width=4.5in}{zhuanjieqishiyitu}

如图\ref{zhuanjieqishiyitu}所示,本课题制作了一个USB转接器,同时作为USB-HID设备的USB主机和PC的USB设备,置于USB-HID 设备与PC之间,将HID 设备的信息读取后进行处理再发给USB主机。

硬件设计工作如下:

\begin{enumerate}
\item 主控芯片选型

由于本课题使用了USBhost 接口和USBdev 接口,所以需要选择带有USB主机(或\gls{OTG})和USB从机接口的单片机。由于作为USB主机的单片机需要实现较多的功能,因此应该选择主频较高的单片机,调查市面上的单片机型号之后,最终确定为STM32F407VG。

\item USB从机芯片选型

作为USB从机的单片机只需要实现USBdev功能即可,所以可以选择价格较低的单片机。考虑到开发工具的统一,最终选定的作为USB 从机接口的单片机是STM32F103C8。


\item PCB设计

由于市面上没有相关产品,本课题需要自行设计PCB并进行焊接。具体的电路设计将在下文中进行介绍。考虑到焊接难度,主控芯片选择的是较容易焊接的QFP 封装。
\end{enumerate}
软件设计工作如下:
\begin{enumerate}
\item 在主控芯片上安装RT-Thread系统

RT-Thread是新兴的国产实时操作系统(RTOS)。在核心板上移植RT-Thread,可以让RT-Thread运行在本课题的硬件上,从而使用RT-Thread带有的系统功能,方便本系统的开发,也能增强系统能力。

\item 编写基于RTT系统的相关模块驱动

由于RT-Thread是嵌入式操作系统,所以操作底层硬件需要编写相应的驱动程序,便于应用层进行调用。

\item 编写应用层程序

应用层程序即功能型程序,就是将本课题需要实现的功能通过编写线程进行实现。


\item 编写Lua脚本程序

由于用户有时候会有自行定制功能的需求,为了实现跨平台,避免使用上位机软件,所以在RT-Thread里面使用了lua组件。编写Lua脚本程序可以作为脚本扩展的示例程序。

\end{enumerate}


\section{预期功能}
本系统由于是为了应用而开发,所以应用层的工作量比例较大,开发出了以下多种具有实用价值的功能:
\begin{enumerate}
\item 改键功能

通过\verb|修改文件系统中的KEY_T键盘映射文件,|可以修改键盘键位映射,语法形式为:“RALT=RCTRL”。这个语句表示右边的ALT 键将映射为右边的CTRL键。各个语句之间并不互相影响,即写入“RCTRL=RALT RCTRL=RCTRL”则相当于没有改键。

\item 键盘宏

键盘宏,顾名思义就是将快捷键定义为其他事件。比如将左边的CTRL+P定义为输出银行卡号码,将右边的ALT+N定义为向下箭头(仿\gls{emacs} 快捷键)。通过使用键盘宏,可以极大地增强用户对键盘的控制能力,减少关键信息的重复输入(手机号码等)。通过仿\gls{emacs} 定义键位,可以实现全局emacs键位,可以极大地提高用户的输入速度。

\item 蓝牙KVM

  \gls{KVM}常用于多电脑同时操作,可以将鼠标键盘视频模拟连接到其他电脑上。本系统只实现了键盘和鼠标的连接。本系统的硬件电路上集成有一个蓝牙模块,可以和其他的相同系统进行蓝牙通信。通过蓝牙,可以使用一套鼠标键盘控制多个硬件,极大地减少了跨平台工作者的操作复杂度。

\item 语音识别

本系统硬件上集成了一个国产语音识别芯片LD3320,可以通过设置拼音进行识别。通过语音识别,可以方便地对电脑进行控制,比如在没有空余手的时候进行语音翻页。



\item Lua脚本

通过RTT的Lua组件,使系统支持Lua脚本编程。Lua脚本置于板上的SPI FLASH中,可以通过虚拟U盘接口在其中写入Lua脚本文件并运行。

\end{enumerate}












