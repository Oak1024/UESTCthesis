% !Mode:: "TeX:UTF-8"

\chapter{硬件设计}
\section{USB接口设计}
\subsection{USB主机接口设计}
\pic[h]{USB主机部分原理图}{width=3in}{USBhost}
如图\ref{USBhost}所示,USBhost部分共有4根线。其中,+5引脚为给USB设备供电用的电源线。D+与D-是USB信号线,使用差分信号和USB 设备进行通信。22欧姆电阻用来做阻抗匹配和短路保护。D信号线通过15K电阻下拉,用于检测USB设备的插入和速度。D+上拉则为高速设备,D- 上拉则为低速设备。两个15K下拉电阻也能防止USB信号线出现浮空引起误判。
  
\subsection{USB从机接口设计}
\pic[h]{USB从机部分原理图}{width=2.5in}{USBdev}
如图\ref{USBdev}所示,USB从机部分共五个引脚,分别为电源、D-、D+、GND、外壳。+5V引脚是PC给系统供电用引脚,经过电源芯片处理后转为3.3V给系统中各种芯片、模块供电。D-、D+信号线连22欧电阻用做阻抗匹配与短路保护。其中,D+信号线通过1.5K电阻接入单片机的USB 控制引脚,这样单片机的控制引脚就可以控制USB设备的接入,实现在初始化完成后接入PC。此处使用1.5K电阻上拉,主机处使用15K 电阻下拉,连接后信号线约为3V,可以被判断为高电平。由于使用USB2.0,全速设备,故在D+处上拉。

\subsection{USB电源设计}
\pic[h]{USB从机部分原理图}{width=3.5in}{USBpower}
如图\ref{USBpower}所示,USB电源部分使用STMPS2141芯片作为电源控制芯片,可以用作短路保护、电源控制。电源芯片输出直接连USB母口供电,输入则为USB公头处理后(短路保护)的5V电源。当电源输出发生短路时FAULT引脚置位,提示单片机电源出现错误。MCU上电并初始化完成后,置位EN引脚,给USB设备通电。FAULT处LED用来提示电源短路,OUT出LED则用于提示设备已通电。

\section{存储电路设计}
\pic[h]{存储电路部分原理图}{width=3.5in}{saving}
如图\ref{saving}所示,存储部分使用W25系列芯片,本设计选用W25Q32,容量为4MB,足够存储大量的脚本文件。FLASH部分采用SPI 接口,可以使用几M的操作速度,因此可以大大加速U盘操作速度。3.3V与GND之间加入104电容用做去耦。由于系统整体电路简单,WP脚直接连入3.3V,根据需求在软件上层决定是否写保护。CS脚使用软件控制,使引脚分配更加灵活。

\section{OLED电路设计}
\pic[h]{OLED电路部分原理图}{width=3.5in}{OLED}
如图\ref{OLED}所示,OLED部分电路使用裸屏OLED,共计30脚,其中大部分为接地引脚。OLED使用3.3V供电,通过SPI接口与MCU进行通信。OLED 像素为128*64,可显示16*4共计64个字母。OLED模块用于显示系统的调试信息与系统状态。

由于系统空间有限,此处输出设备使用裸屏OLED,使用FPC与主板相连。

与传统SPI接口相区别,此处OLED模块多出一个DC脚,用于选择数据与命令。发送命令时,将DC脚拉低,然后发送命令内容。发送数据时则先将DC脚拉高再发送数据。

为增大布线灵活性,此处同样使用普通IO作为CS控制引脚。

\section{USB设备接口单片机电路设计}
\pic[h]{USB设备接口单片机电路部分原理图}{width=5in}{USBMCU}
如图\ref{USBMCU}所示,USB设备接口使用STM32F103C8单片机。这款单片机是带有USB设备接口的较廉价的单片机之一。该单片机内置64KBFLASH,20KBRAM,足以用作USB设备接口电路。

单片机5、6脚接入8M晶振,经内部倍频后转为72M信号用做系统时钟。片上串口IO用作串口功能,从板上引出,用做调试。BOOT0引脚直接拉低,表示直接从内部FLASH启动。下载接口使用两线制SWD接口。SWD接口相比JTAG接口更稳定,速度更快,占用IO更少。下载时使用Jlink仿真器,经测试可使用4M速度进行程序下载。

\section{USB主机单片机电路设计}
\pic[h]{USB主机单片机电路部分原理图}{width=5in}{USBMCU2}

如图\ref{USBMCU2}所示,USB主机芯片采用STM32F405RG单片机(同时也作为主控单片机),片上含一个OTG接口,内置1MBFLASH与192KRAM。

单片机的晶振部分同样采用8M晶振,倍频后使用168M内部时钟。由于本系统应用层语法分析器功能复杂,故选用STM32F4系列高速单片机进行处理,以减小延迟感。

单片机的串口部分用做波特率115200的串口功能方便引出调试。




