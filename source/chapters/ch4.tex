% !Mode:: "TeX:UTF-8"

\chapter{软件设计}

\section{线程管理}

\subsection{线程功能介绍}

\subsubsection{USBhost接口监测线程}
定时根据USB状态,对USBhost接口上的数据进行处理,读取USB键盘的数据。

\subsubsection{MCU间通信线程}
这个线程主要用于处理两个单片机之间的通信问题。由于两个单片机的任务不同,程序结构差异较大,所以需要专门开启一个线程,
与作为USBdev接口的单片机进行通信,发送与接收U盘的读写、键盘的按键信息。

\subsubsection{FLASH控制线程}
由于两个单片机共用一个FLASH芯片,而且读写FLASH需要的时间较长,因此需要使用一个线程用来监测通信接口的IO信号,对FLASH进行相应读写操作。

\subsubsection{应用层线程}
这个线程主要用来初始化应用层程序,对配置文件进行语法分析并执行。

\subsection{线程间通信量、同步量}
\subsubsection{消息序列1}
用于应用层线程与USBhost接口线程进行通信,作为按键存储的FIFO。
\subsubsection{消息序列2}
用于通信线程与应用层线程进行通信,作为按键存储的FIFO。
\subsubsection{信号量1}
用于FLASH读写的同步,保持只有一个线程在操作FLASH 芯片。
\subsubsection{信号量2}
用于通信线程与FLASH控制线程间的同步。

\subsection{线程间通信量、同步量}

\section{软件流程}
\subsection{USB接口监测线程}
USB接口线程在开机后便初始化,初始化完成后不断对USB信息的进行读取。一旦读取到USB传来的有效信息,就将其打包后发送给应用层线程。

\subsection{通信线程}
\pic[h]{通信线程流程图}{}{liu2}
如图\ref{liu2}所示,通信线程首先通过IO口操作和USB接口单片机进行通信,然后阻塞在消息序列的等待上。一旦收到消息,便根据具体情况把消息发送给接口单片机。

\subsection{应用层控制线程}
\pic[h]{应用层线程流程图}{}{liu4}
如图\ref{liu4}所示,应用层线程在启动后会使用FATFS对FLASH内的脚本文件进行一次读取。读取后通过自制的语法分析器进行分析,并生成相应的过滤器链表。之后此进程阻塞在按键消息的等待上,并处理得到的按键消息。

\section{信息流向}
\pic[h]{信息流程示意图}{}{liu5}
如图\ref{liu5}所示,USB信息从USB母头传至USB公头共经历3个FIFO。信息从母头进入后首先被USB接口线程检测到,经过打包后发入第一个FIFO。此时阻塞在此FIFO的线程——应用层线程就绪并执行,根据FLASH中脚本内容进行过滤并执行脚本内容,将脚本对应的按键根据脚本规则发入第二个FIFO。此时阻塞在第二个FIFO的通信线程就绪并执行,将收到的信息打包发给USB接口单片机。USB接口单片机中通过环结构对外界信息进行存储,在端点空闲后将信息发送给电脑。

\section{语法分析}
本设计的脚本使用精简版的AHK脚本语言(暂命名为BMK 语言)。目前可以实现复杂快捷键过滤、发送长名称键、语法错误分析等功能。
\subsection{快捷键语法介绍}
\verb|BMK语言的快捷键过滤器部分使用不常用的四个符号代替具体快捷键"CTRL"-'^' ,"WIN"-'#',"ALT"-'^',"SHIFT"-'+'。通过这几个符号可以代替快捷键全称,极大地简化了脚本的编写。另外,使用'>'和'<'符号来确定快捷键的左右方向,缺省值为双向。'<'为仅限左边的键,'>'为仅限右边的键。例如:"<+>!q"即为左shift+右alt键+q组合快捷键。|

\subsection{快捷键语法介绍}
\verb|一个完整的BMK脚本是由脚本段组成的。每个脚本段由三部分构成:1快捷键过滤器,2双冒号分隔符,3:按键脚本。示例:>!n::{down}。意为按下右alt+n 快捷键,可以发送给PC“down”键(即向下箭头)(这个语法段是emacs的一个快捷键操作)。|

每个语法段中的三个部分缺一不可,每部分都有各自的作用。
\begin{enumerate}
\item 快捷键过滤器
抓取快捷键,提供脚本执行情景。

\item 双冒号分隔符
分割过滤器与按键脚本,防止混杂出现语法错误。

\item 按键脚本
作为脚本的执行部分,必不可少。

\item 其他
\verb|使用';'为行开头可以进行注释,不区分大小写。|
\end{enumerate}
\verb|按键脚本中,普通按键直接书写相应ASCii码作为表示。长按键如“TAB”、“SPACE”等,使用大括号括起表示,如“{tab}”|。
\subsection{语法分析器简介}
\subsubsection{分段器}
分段器将程序脚本根据0x0A标志分为多个语法段,并去掉相应的0x0D标志。每个程序段构成一个独立的语法段落,实现一个独立的快捷键过滤并执行脚本的功能。
\subsubsection{语法分析器}
语法分析器将token分析器得到的token序列进行分析,并在程序段结束时对整个程序段进行注册。语法分析器可以根据token序列的具体内容,判断语法的正确与否。比如在声明一组快捷键过滤器后又进行了一组快捷键的声明,这时会提示快捷键声明重复。
\subsubsection{token分析器}
token分析器将分段器分出的程序段看做纯ASCii序列,然后整理成token流。token分析器将token分为单侧控制键、双侧控制键、其他键、脚本键等几种,并根据分析器的状态和当前ASCii码将ASCii流根据情况整理为token流。分析器还可以发现错误token,生成提示。

\section{lua脚本}
\subsection{使用lua脚本的原因}
Lua 是一个小巧的脚本语言。是巴西里约热内卢天主教大学里的一个研究小组,由Roberto Ierusalimschy、Waldemar Celes 和 Luiz Henrique de Figueiredo所组成并于1993年开发。 其设计目的是为了嵌入应用程序中,从而为应用程序提供灵活的扩展和定制功能。Lua由标准C编写而成,几乎在所有操作系统和平台上都可以编译,运行。

由于用户有时候会有自行定制功能的需求,为了实现跨平台,避免使用上位机软件,所以在RT-Thread里面使用了lua组件。通过使用lua 组件,可以使用户能在转接器的文件系统中编写脚本文件,进而通过脚本文件对整个系统进行控制,从而能够轻松地给系统添加功能,极大地增强了系统的可定制化程度。
\subsection{lua脚本源码}
下面是一段示例程序,功能是给系统加入宏录制功能。

\noindent
\ttfamily
\hlstd{}\hllin{01\ }\hlkwa{function\ }\hlstd{}\hlkwd{key\textunderscore handle}\hlstd{}\hlopt{(}\hlstd{data}\hlopt{)}\\
\hllin{02\ }\hlstd{}\hlstd{\ \ \ \ }\hlstd{}\hlkwa{if\ }\hlstd{macro\textunderscore flag\ }\hlkwa{then}\\
\hllin{03\ }\hlstd{}\hlstd{\ \ \ \ \ \ \ \ }\hlstd{}\hlkwa{for\ }\hlstd{i\ }\hlopt{=}\hlstd{}\hlnum{1}\hlstd{}\hlopt{,}\hlstd{}\hlnum{10\ }\hlstd{}\hlkwa{do}\\
\hllin{04\ }\hlstd{}\hlstd{\ \ \ \ \ \ \ \ \ \ \ \ }\hlstd{key\textunderscore data}\hlopt{{[}}\hlstd{key\textunderscore index}\hlopt{{]}=}\hlstd{data}\hlopt{{[}}\hlstd{i}\hlopt{{]};}\\
\hllin{05\ }\hlstd{}\hlstd{\ \ \ \ \ \ \ \ \ \ \ \ }\hlstd{key\textunderscore index}\hlopt{=}\hlstd{key\textunderscore index}\hlopt{+}\hlstd{}\hlnum{1}\hlstd{}\hlopt{;}\\
\hllin{06\ }\hlstd{}\hlstd{\ \ \ \ \ \ \ \ }\hlstd{}\hlkwa{end}\\
\hllin{07\ }\hlstd{}\hlstd{\ \ \ \ }\hlstd{}\hlkwa{end}\\
\hllin{08\ }\hlstd{}\hlstd{\ \ \ \ }\hlstd{}\hlkwb{print}\hlstd{}\hlopt{(}\hlstd{}\hlstr{"}\hlstd{\ \ \ }\hlstr{key}\hlstd{\ \ }\hlstr{"}\hlstd{}\hlopt{..}\hlstd{key\textunderscore index}\hlopt{..}\hlstd{}\hlstr{"}\hlesc{$\backslash$n}\hlstr{"}\hlstd{}\hlopt{);}\\
\hllin{09\ }\hlstd{}\hlkwa{end}\\
\hllin{10\ }\hlstd{}\\
\hllin{11\ }\hlkwa{function\ }\hlstd{}\hlkwd{macro\textunderscore play}\hlstd{}\hlopt{(}\hlstd{data}\hlopt{)}\hlstd{\ \ }\hlopt{}\\
\hllin{12\ }\hlstd{}\hlstd{\ \ \ \ }\hlstd{}\hlkwd{flag\textunderscore set}\hlstd{}\hlopt{(}\hlstd{}\hlnum{2}\hlstd{}\hlopt{,}\hlstd{}\hlnum{0}\hlstd{}\hlopt{);\ }\\
\hllin{13\ }\hlstd{}\hlstd{\ \ \ \ }\hlstd{}\hlkwb{print}\hlstd{}\hlopt{(}\hlstd{}\hlstr{"macro\textunderscore play}\hlesc{$\backslash$n}\hlstr{"}\hlstd{}\hlopt{)\ ;}\\
\hllin{14\ }\hlstd{}\hlstd{\ \ \ \ }\hlstd{}\hlkwa{local\ }\hlstd{one}\hlopt{=}\hlstd{}\hlnum{0}\hlstd{}\hlopt{;}\\
\hllin{15\ }\hlstd{}\hlstd{\ \ \ \ }\hlstd{}\hlkwa{for\ }\hlstd{i\ }\hlopt{=}\hlstd{}\hlnum{1}\hlstd{}\hlopt{,}\hlstd{key\textunderscore index}\hlopt{/}\hlstd{}\hlnum{10}\hlstd{}\hlopt{{-}}\hlstd{}\hlnum{1\ }\hlstd{}\hlkwa{do}\\
\hllin{16\ }\hlstd{\\
\hllin{17\ }}\hlstd{\ \ \ \ \ \ \ \ }\hlstd{}\hlkwd{key\textunderscore put\textunderscore pure}\hlstd{}\hlopt{(}\hlstd{key\textunderscore data}\hlopt{{[}(}\hlstd{i}\hlopt{{-}}\hlstd{}\hlnum{1}\hlstd{}\hlopt{){*}}\hlstd{}\hlnum{10}\hlstd{}\hlopt{+}\hlstd{}\hlnum{1}\hlstd{}\hlopt{{]},}\\
\hllin{18\ }\hlstd{}\hlstd{\ \ \ \ \ \ \ \ \ \ \ \ }\hlstd{key\textunderscore data}\hlopt{{[}(}\hlstd{i}\hlopt{{-}}\hlstd{}\hlnum{1}\hlstd{}\hlopt{){*}}\hlstd{}\hlnum{10}\hlstd{}\hlopt{+}\hlstd{}\hlnum{2}\hlstd{}\hlopt{{]},}\\
\hllin{19\ }\hlstd{}\hlstd{\ \ \ \ \ \ \ \ \ \ \ \ }\hlstd{key\textunderscore data}\hlopt{{[}(}\hlstd{i}\hlopt{{-}}\hlstd{}\hlnum{1}\hlstd{}\hlopt{){*}}\hlstd{}\hlnum{10}\hlstd{}\hlopt{+}\hlstd{}\hlnum{3}\hlstd{}\hlopt{{]},}\\
\hllin{20\ }\hlstd{}\hlstd{\ \ \ \ \ \ \ \ \ \ \ \ }\hlstd{key\textunderscore data}\hlopt{{[}(}\hlstd{i}\hlopt{{-}}\hlstd{}\hlnum{1}\hlstd{}\hlopt{){*}}\hlstd{}\hlnum{10}\hlstd{}\hlopt{+}\hlstd{}\hlnum{4}\hlstd{}\hlopt{{]},}\\
\hllin{21\ }\hlstd{}\hlstd{\ \ \ \ \ \ \ \ \ \ \ \ }\hlstd{key\textunderscore data}\hlopt{{[}(}\hlstd{i}\hlopt{{-}}\hlstd{}\hlnum{1}\hlstd{}\hlopt{){*}}\hlstd{}\hlnum{10}\hlstd{}\hlopt{+}\hlstd{}\hlnum{5}\hlstd{}\hlopt{{]},}\\
\hllin{22\ }\hlstd{}\hlstd{\ \ \ \ \ \ \ \ \ \ \ \ }\hlstd{key\textunderscore data}\hlopt{{[}(}\hlstd{i}\hlopt{{-}}\hlstd{}\hlnum{1}\hlstd{}\hlopt{){*}}\hlstd{}\hlnum{10}\hlstd{}\hlopt{+}\hlstd{}\hlnum{6}\hlstd{}\hlopt{{]},}\\
\hllin{23\ }\hlstd{}\hlstd{\ \ \ \ \ \ \ \ \ \ \ \ }\hlstd{key\textunderscore data}\hlopt{{[}(}\hlstd{i}\hlopt{{-}}\hlstd{}\hlnum{1}\hlstd{}\hlopt{){*}}\hlstd{}\hlnum{10}\hlstd{}\hlopt{+}\hlstd{}\hlnum{7}\hlstd{}\hlopt{{]},}\\
\hllin{24\ }\hlstd{}\hlstd{\ \ \ \ \ \ \ \ \ \ \ \ }\hlstd{key\textunderscore data}\hlopt{{[}(}\hlstd{i}\hlopt{{-}}\hlstd{}\hlnum{1}\hlstd{}\hlopt{){*}}\hlstd{}\hlnum{10}\hlstd{}\hlopt{+}\hlstd{}\hlnum{8}\hlstd{}\hlopt{{]},}\\
\hllin{25\ }\hlstd{}\hlstd{\ \ \ \ \ \ \ \ \ \ \ \ }\hlstd{key\textunderscore data}\hlopt{{[}(}\hlstd{i}\hlopt{{-}}\hlstd{}\hlnum{1}\hlstd{}\hlopt{){*}}\hlstd{}\hlnum{10}\hlstd{}\hlopt{+}\hlstd{}\hlnum{9}\hlstd{}\hlopt{{]},}\\
\hllin{26\ }\hlstd{}\hlstd{\ \ \ \ \ \ \ \ \ \ \ \ }\hlstd{key\textunderscore data}\hlopt{{[}(}\hlstd{i}\hlopt{{-}}\hlstd{}\hlnum{1}\hlstd{}\hlopt{){*}}\hlstd{}\hlnum{10}\hlstd{}\hlopt{+}\hlstd{}\hlnum{10}\hlstd{}\hlopt{{]}}\\
\hllin{27\ }\hlstd{}\hlstd{\ \ \ \ \ \ \ \ \ \ \ \ }\hlstd{}\hlopt{);}\\
\hllin{28\ }\hlstd{}\hlstd{\ \ \ \ }\hlstd{}\hlkwa{end}\\
\hllin{29\ }\hlstd{}\hlkwa{end\ }\\
\hllin{30\ }\hlstd{}\\
\hllin{31\ }\hlkwa{function\ }\hlstd{}\hlkwd{macro\textunderscore end}\hlstd{}\hlopt{(}\hlstd{data}\hlopt{)\ }\\
\hllin{32\ }\hlstd{}\hlstd{\ \ \ \ }\hlstd{}\hlkwb{print}\hlstd{}\hlopt{(}\hlstd{}\hlstr{"macro\textunderscore end}\hlesc{$\backslash$n}\hlstr{"}\hlstd{}\hlopt{);}\\
\hllin{33\ }\hlstd{}\hlstd{\ \ \ \ }\hlstd{}\hlkwd{flag\textunderscore set}\hlstd{}\hlopt{(}\hlstd{}\hlnum{0}\hlstd{}\hlopt{,}\hlstd{}\hlnum{0}\hlstd{}\hlopt{);}\hlstd{\ \ \ \ \ \ \ \ \ \ }\hlopt{}\hlstd{}\hlslc{{-}{-}keyboard\ disable}\\
\hllin{34\ }\hlstd{}\hlkwa{end\ }\\
\hllin{35\ }\hlstd{}\\
\hllin{36\ }\hlkwa{function\ }\hlstd{}\hlkwd{macro\textunderscore start}\hlstd{}\hlopt{(}\hlstd{data}\hlopt{)}\hlstd{\ \ }\hlopt{}\\
\hllin{37\ }\hlstd{}\hlstd{\ \ \ \ }\hlstd{}\hlkwb{print}\hlstd{}\hlopt{(}\hlstd{}\hlstr{"macro\textunderscore start}\hlesc{$\backslash$n}\hlstr{"}\hlstd{}\hlopt{);}\\
\hllin{38\ }\hlstd{}\hlstd{\ \ \ \ }\hlstd{}\hlkwa{if\ }\hlstd{macro\textunderscore flag\ }\hlkwa{then}\\
\hllin{39\ }\hlstd{}\hlstd{\ \ \ \ \ \ \ \ }\hlstd{macro\textunderscore flag}\hlopt{=}\hlstd{}\hlkwa{false}\hlstd{}\hlopt{;}\\
\hllin{40\ }\hlstd{}\hlstd{\ \ \ \ \ \ \ \ }\hlstd{}\hlkwd{macro\textunderscore end}\hlstd{}\hlopt{(}\hlstd{data}\hlopt{);}\\
\hllin{41\ }\hlstd{}\hlstd{\ \ \ \ \ \ \ \ }\hlstd{}\hlkwa{return\ }\\
\hllin{42\ }\hlstd{}\hlstd{\ \ \ \ }\hlstd{}\hlkwa{else}\\
\hllin{43\ }\hlstd{}\hlstd{\ \ \ \ \ \ \ \ }\hlstd{key\textunderscore index}\hlopt{=}\hlstd{}\hlnum{1}\hlstd{}\hlopt{;}\\
\hllin{44\ }\hlstd{}\hlstd{\ \ \ \ \ \ \ \ }\hlstd{}\hlkwd{flag\textunderscore set}\hlstd{}\hlopt{(}\hlstd{}\hlnum{0}\hlstd{}\hlopt{,}\hlstd{}\hlnum{1}\hlstd{}\hlopt{);}\hlstd{\ \ \ \ \ \ }\hlopt{}\hlstd{}\hlslc{{-}{-}keyboard\ enable}\\
\hllin{45\ }\hlstd{}\hlstd{\ \ \ \ \ \ \ \ }\hlstd{macro\textunderscore flag}\hlopt{=}\hlstd{}\hlkwa{true}\hlstd{}\hlopt{;}\\
\hllin{46\ }\hlstd{}\hlstd{\ \ \ \ }\hlstd{}\hlkwa{end}\\
\hllin{47\ }\hlstd{\\
\hllin{48\ }}\hlstd{\ \ \ \ }\hlstd{}\\
\hllin{49\ }\hlkwa{end\ }\\
\hllin{50\ }\hlstd{}\\
\hllin{51\ }\hlslc{{-}{-}main}\\
\hllin{52\ }\hlstd{key\textunderscore data}\hlopt{=\{}\hlstd{n}\hlopt{=}\hlstd{}\hlnum{300}\hlstd{}\hlopt{\};}\\
\hllin{53\ }\hlstd{key\textunderscore index}\hlopt{=}\hlstd{}\hlnum{1}\hlstd{}\hlopt{;}\\
\hllin{54\ }\hlstd{macro\textunderscore flag}\hlopt{=}\hlstd{}\hlkwa{false}\hlstd{}\hlopt{;}\\
\hllin{55\ }\hlstd{event}\hlopt{=\{}\hlstd{key\textunderscore handle}\hlopt{,}\hlstd{mouse\textunderscore handle}\hlopt{,}\hlstd{macro\textunderscore start}\hlopt{,}\hlstd{macro\textunderscore play}\hlopt{\}}\\
\hllin{56\ }\hlstd{}\\
\hllin{57\ }\hlkwd{key\textunderscore register}\hlstd{}\hlopt{(}\hlstd{}\hlnum{3}\hlstd{}\hlcom{{-}{-}{[}{[}macro\textunderscore start{]}{]}}\hlstd{}\hlopt{,}\hlstd{}\hlstr{'o'}\hlstd{}\hlopt{,}\hlstd{}\hlstr{"rctrl"}\hlstd{}\hlopt{)}\\
\hllin{58\ }\hlstd{}\hlkwd{key\textunderscore register}\hlstd{}\hlopt{(}\hlstd{}\hlnum{4}\hlstd{}\hlcom{{-}{-}{[}{[}macro\textunderscore start{]}{]}}\hlstd{}\hlopt{,}\hlstd{}\hlstr{'i'}\hlstd{}\hlopt{,}\hlstd{}\hlstr{"rctrl"}\hlstd{}\hlopt{)}\\
\hllin{59\ }\hlstd{}\\
\hllin{60\ }\hlkwa{while\ }\hlstd{}\hlnum{1\ }\hlstd{}\hlkwa{do\ }\\
\hllin{61\ }\hlstd{}\hlstd{\ \ \ \ }\hlstd{a}\hlopt{=\{}\hlstd{}\hlkwd{wait\textunderscore event}\hlstd{}\hlopt{()\};}\\
\hllin{62\ }\hlstd{}\hlstd{\ \ \ \ }\hlstd{}\hlopt{(}\hlstd{event}\hlopt{{[}}\hlstd{a}\hlopt{{[}}\hlstd{}\hlnum{10}\hlstd{{]}{]}}\hlopt{)(}\hlstd{a}\hlopt{);}\\
\hllin{63\ }\hlstd{}\hlkwa{end\ }\hlstd{}\\
\mbox{}
\normalfont
\normalsize






