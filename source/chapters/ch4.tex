% !Mode:: "TeX:UTF-8"

\chapter{软件设计}

\section{线程管理}

\subsection{线程功能介绍}

\subsubsection{USBhost接口监测线程}
定时根据USB状态,对USBhost接口上的数据进行处理,读取USB键盘的数据。

\subsubsection{MCU间通信线程}
与作为USBdev接口的单片机进行通信,发送与接收U盘的读写、键盘的按键信息。

\subsubsection{FLASH控制线程}
监测通信接口的IO信号,对FLASH进行相应读写操作。

\subsubsection{应用层线程}
对配置文件进行语法分析并执行。

\subsection{线程间通信量、同步量}
\subsubsection{消息序列1}
用于应用层线程与USBhost接口线程进行通信,作为按键存储的FIFO。
\subsubsection{消息序列2}
用于通信线程与应用层线程进行通信,作为按键存储的FIFO。
\subsubsection{信号量1}
用于FLASH读写的同步,保持只有一个线程在操作FLASH芯片。
\subsubsection{信号量2}
用于通信线程与FLASH控制线程间的同步。

\subsection{线程间通信量、同步量}
\pic[h]{线程间关系}{width=4in}{xiancheng}
图\ref{xiancheng}为系统中各线程间的同步关系。第一层为应用层线程,没有任何硬件操作,第二层为硬件层线程,和硬件操作紧密相关。各线程间紧密相关,通过各种通信量、同步量进行协作。

\section{软件流程}
\subsection{USB接口监测线程}
USB接口线程在开机后便初始化,初始化完成后不断对USB信息的进行读取。一旦读取到USB传来的有效信息,就将其打包后发送给应用层线程。

\subsection{通信线程}
\pic[h]{通信线程流程图}{}{liu2}
如图\ref{liu2}所示,通信线程首先通过IO口操作和USB接口单片机进行通信,然后阻塞在消息序列的等待上。一旦收到消息,便根据具体情况把消息发送给接口单片机。

\subsection{应用层控制线程}
\pic[h]{应用层线程流程图}{}{liu4}
如图\ref{liu4}所示,应用层线程在启动后会使用FATFS对FLASH内的脚本文件进行一次读取。读取后通过自制的语法分析器进行分析,并生成相应的过滤器链表。之后此进程阻塞在按键消息的等待上,并处理得到的按键消息。

\section{信息流向}
\pic[h]{信息流程示意图}{}{liu5}
如图\ref{liu5}所示,USB信息从USB母头传至USB公头共经历3个FIFO。信息从母头进入后首先被USB接口线程检测到,经过打包后发入第一个FIFO。此时阻塞在此FIFO的线程——应用层线程就绪并执行,根据FLASH中脚本内容进行过滤并执行脚本内容,将脚本对应的按键根据脚本规则发入第二个FIFO。此时阻塞在第二个FIFO的通信线程就绪并执行,将收到的信息打包发给USB接口单片机。USB接口单片机中通过环结构对外界信息进行存储,在端点空闲后将信息发送给电脑。

\section{语法分析}
本设计的脚本使用精简版的AHK脚本语言(暂命名为BMK语言)。目前可以实现复杂快捷键过滤、发送长名称键、语法错误分析等功能。
\subsection{快捷键语法介绍}
\verb|BMK语言的快捷键过滤器部分使用不常用的四个符号代替具体快捷键"CTRL"-'^' ,"WIN"-'#',"ALT"-'^',"SHIFT"-'+'。通过这几个符号可以代替快捷键全称,极大地简化了脚本的编写。另外,使用'>'和'<'符号来确定快捷键的左右方向,缺省值为双向。'<'为仅限左边的键,'>'为仅限右边的键。例如:"<+>!q"即为左shift+右alt键+q组合快捷键。|

\subsection{快捷键语法介绍}
\verb|一个完整的BMK脚本是由脚本段组成的。每个脚本段由三部分构成:1快捷键过滤器,2双冒号分隔符,3:按键脚本。示例:>!n::{down}。意为按下右alt+n快捷键,可以发送给PC“down”键(即向下箭头)(这个语法段是emacs的一个快捷键操作)。|

每个语法段中的三个部分缺一不可,每部分都有各自的作用。
\begin{enumerate}
\item 快捷键过滤器
抓取快捷键,提供脚本执行情景。

\item 双冒号分隔符
分割过滤器与按键脚本,防止混杂出现语法错误。

\item 按键脚本
作为脚本的执行部分,必不可少。

\item 其他
\verb|使用';'为行开头可以进行注释,不区分大小写。|
\end{enumerate}
\verb|按键脚本中,普通按键直接书写相应ASCii码作为表示。长按键如“TAB”、“SPACE”等,使用大括号括起表示,如“{tab}”|。
\subsection{语法分析器简介}
\subsubsection{分段器}
分段器将程序脚本根据0x0A标志分为多个语法段,并去掉相应的0x0D标志。每个程序段构成一个独立的语法段落,实现一个独立的快捷键过滤并执行脚本的功能。
\subsubsection{语法分析器}
语法分析器将token分析器得到的token序列进行分析,并在程序段结束时对整个程序段进行注册。语法分析器可以根据token序列的具体内容,判断语法的正确与否。比如在声明一组快捷键过滤器后又进行了一组快捷键的声明,这时会提示快捷键声明重复。
\subsubsection{token分析器}
token分析器将分段器分出的程序段看做纯ASCii序列,然后整理成token流。token分析器将token分为单侧控制键、双侧控制键、其他键、脚本键等几种,并根据分析器的状态和当前ASCii码将ASCii流根据情况整理为token流。分析器还可以发现错误token,生成提示。







