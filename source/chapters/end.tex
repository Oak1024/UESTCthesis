% !Mode:: "TeX:UTF-8"

\chapter{结束语}
\section{全文总结}
软件跨平台一直是计算机软件的一个热点和难题。对于键盘控制软件,跨平台更是重中之重。本课题通过使用硬件实现键盘控制,巧妙地避开了软件跨平台的问题,实现了一个跨平台的智能USB转接器。

本文首先介绍了USB转接器的背景与意义,然后深入阐述了USB转接器的主控芯片选型、PCB设计以及系统安装等设计内容,并且给出了系统的示意图,形象地展示了USB-HID设备、转接器与PC之间的连接关系。

然后,本文介绍了USB转接器的整体方案设计。通过介绍USB与嵌入式实时系统(RTOS),引出了对国产实时操作系统RT-Thread的介绍。通过对RT-Thread进行介绍,解释为何使用RT-Thread用做系统的操作系统。

之后,本文介绍了本课题的硬件设计与软件设计。在硬件设计的介绍中,本文通过展示系统各个模块的原理图,解释了硬件电路的原理和各个模块的作用,阐述了各个模块之间的联系。在软件设计的介绍中,首先介绍了线程、同步量的作用,进而引出各个线程的流程,并通过流程图形象地描述了线程的工作原理。
最后,通过介绍lua语言脚本并展示lua语言脚本插件的源码,描述了本系统极高的可定制程度。

\section{特色与创新}
本系统基于本人需求进行开发,市面上并没有相同的产品出售,因此系统的绝大部分功能都是自主创新。
\begin{enumerate}
\item 嵌入式操作系统

本系统基于RT-Thread开发,可扩展性强,可定制性强,且具有多进程处理能力。传统小型嵌入式设备一般直接进行裸机编程,而本系统为了增强系统的能力,基于RT-Thread进行开发,充分利用了RTOS的优势。
\item 在线编程

使用自制脚本分析器,可以进行在线编程,无需仿真器或设备。使得普通用户也能通过修改文件系统内的文件进行复杂的操作,大大增强了系统的可定制程度。
\item 跨平台


通过实现标准USB-HID设备,实现真正的跨平台,Android、MAC、Linux、Windows都通过了测试。

\item 脚本定制

通过使用RT-Thread的lua组件,实现了运行脚本的功能,极大地增强了系统的可定制程度。
\end{enumerate}


\section{未来工作展望}
随着芯片技术的发展,芯片的集成度越来越高,电路板需要的面积逐渐减小。为了增强系统能力,方便插件开发,本系统以后会考虑使用Linux系统,充分利用Linux系统的强大功能来实现键盘增强的功能。
与此同时,越来越多的脚本语言被发明出来。脚本语言可以解释执行,无须编译,而且与硬件底层联系较少,语法限制少,便于学习,所以非常适合用于开发插件。因此,在未来本系统会尝试加入Javascript、Python等脚本语言的解释器,使脚本插件的开发更加简便。





