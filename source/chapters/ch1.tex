% !Mode:: "TeX:UTF-8"

\chapter{引言}
\section{USB转接器的背景与意义}
\newacronym[description=通用串行总线]{USB}{USB}{Universal Serial Bus}
\newacronym[description=实时操作系统]{RTOS}{RTOS}{Real Time Operating System}
\newacronym[description=人机接口设备]{HID}{HID}{Human Interface Device}
\newacronym[description=苹果电脑]{Mac}{Mac}{Macintosh}
\newacronym[description=Linux操作系统]{Linux}{Linux}{Linux is not Unix}
\newacronym[description=串行外设接口]{SPI}{SPI}{Serial Peripheral Interface}
\newacronym[description=RT-Thread实时操作系统]{RTT}{RTT}{RT-Thread}
\newacronym[description=On-The-Go接口]{OTG}{OTG}{On-The-Go}
\newacronym[description=宏文本编辑器]{emacs}{emacs}{Editor MACroS}
\newacronym[description=KVM技术]{KVM}{KVM}{Keyboard Video Mouse}
\newacronym[description=麦克风]{MIC}{MIC}{Microphone}
键盘是最常见的计算机输入设备,它广泛应用于微型计算机和各种终端设备上,计算机操作者通过键盘向计算机输入各种指令、数据,指挥计算机的工作。计算机的运行情况输出到显示器,操作者可以很方便地利用键盘和显示器与计算机对话,对程序进行修改、编辑,控制和观察计算机的运行。

由于键盘是计算机系统的最主要输入设备,对键盘的完全掌控就显得非常重要。由于缺少对键盘的完全掌控,人们往往需要重复输入很多数据,进行重复劳动,极大地浪费了生产力。因此,本系统通过实现一个USB转接器作为键盘的扩展,为未来键盘的发展提供了一个可行的方向。由于USB是键盘的常用接口,因此本课题基于USB接口进行开发。

\gls{USB},是一个外部总线标准,用于规范电脑与外部设备的连接和通讯,是应用在PC领域的接口技术。USB接口支持设备的即插即用和热插拔功能。\gls{USB}是在1994年底由英特尔、康柏、IBM、Microsoft等多家公司联合提出的。从1994年11月11日发表了USB V0.7 版本以后,USB版本经历了多年的发展,已经发展为3.1版本,成为二十一世纪二十年代电脑中的标准扩展接口。

由于键盘作为\gls{HID}设备需要较快的反应速度,所以本课题选用\gls{RTOS}作为主控芯片的操作系统。\gls{RTOS},即实时操作系统,是指当外界事件或数据产生时,能够接受并以足够快的速度予以处理,其处理的结果又能在规定的时间之内来控制生产过程或对处理系统做出快速响应,并控制所有实时任务协调一致运行的操作系统。


\section{国内外的研究现状}
现代键盘和几十年前的键盘相比大部分没有太大区别。少有的几种升级如下:
\begin{enumerate}
\item 蓝牙接口

以前的键盘的接口有AT接口、PS/2接口和最新的USB接口,台式机曾多采用PS/2接口,大多数主板都提供PS/2 键盘接口。而较老的主板常常提供AT 接口也被称为“大口”,已经不常见了。USB作为新型的接口,一些公司迅速推出了USB接口的键盘。由于USB 接口具有热插拔、功能多、速度快等特点,现代电脑普遍使用USB外接键盘。蓝牙接口相比传统键盘,具有无线、可配置等优秀特点,因此被大量用于台式机、平板电脑等不带键盘的电脑设备中。\pic[h]{新型蓝牙键盘}{width=3.5in}{keyboard1}



\item 触摸板

触摸板是一种在平滑的触控板上,利用手指的滑动操作来移动游标的输入装置。当使用者的手指接近触摸板时会使电容量改变,触摸板自身会检测出电容改变量,转换成坐标。其优点在于使用范围较广,全内置、超轻薄笔记本均适用,而且耗电量少,可以提供手写输入功能。随着笔记本的普及,触摸板逐渐被人们所接受,因此出现了带有触摸板的键盘。触摸板通过构造USB复合设备,使一个USB 设备可以拥有触摸板、键盘这两种\gls{HID}功能。\pic[h]{触摸板键盘}{width=3.5in}{keyboard2}

\item 功能型键盘

功能型键盘的特点是与其他键盘相比多出很多标准之外的键,比如:前进后退键、计算器键、收藏夹键等。功能型键盘的自带功能一般通过安装厂商提供的驱动进行实现。如果没有驱动程序,往往只能实现标准的功能。然而大多数厂商都只提供windows系统下的驱动程序,因此其他平台的用户往往无法使用功能型键盘的全部功能。\pic[h]{功能型键盘}{width=3.5in}{keyboard2}


\end{enumerate}

以上几种键盘都是在原有的键盘的基础上实现了小幅度的改进,增加了一些小功能,在提高用户对键盘的掌控力方面并没有太大的提升,没有让用户从重复劳动中解脱出来,也没用给予用户定制功能的能力。本系统则希望通过嵌入式技术改变这一现状——通过实现一个USB转接器增强了用户对键盘的掌控力,真正让用户从计算机输入的重复劳动中解脱出来,从而极大地扩展键盘的功能。

\section{理论依据}

\gls{USB}作为一种总线标准,已经逐渐被近些年新生产的ARM内核单片机实现。因此,通过单片机对USB设备进行控制与通过单片机实现USB设备都成为了可能。此外,单片机与单片机之间,可以方便地通过\gls{SPI}接口进行通信。综上,可以通过两个单片机分别作为USB主机与USB 设备,单片机之间通过\gls{SPI}接口进行连接,实现一个可以进行USB信息转发的USB转接器。

USB协议中,有一种设备被称为“复合设备”,可以在USB主机端枚举为一个设备,但实现多个接口。通过实现一个USB复合设备,可以使USB转接器在USB主机上枚举为鼠标、键盘、U盘三个设备,从而使得USB转接器支持USB鼠标与键盘的转接,同时可以通过实现U盘来提供配置文件的写入接口,便于对配置文件进行修改。

Lua 是一个小巧的脚本语言,可以通过嵌入其他应用程序,为其他程序提供灵活的扩展与定制功能。为了对USB转接器进行灵活的配置,本课题选用Lua语言作为转接器的脚本配置语言,期望通过在U盘上写入Lua语言文件,实现在线编程的功能。

\gls{RTT}是一款主要由中国开源社区主导开发的开源实时操作系统(许可证GPLv2)。实时线程操作系统不仅仅是一个单一的实时操作系统内核,它也是一个完整的应用系统,包含了实时、嵌入式系统相关的各个组件:TCP/IP协议栈,文件系统,libc接口,图形用户界面等。由于\gls{RTT}具有开源、免费、组件完善等特点,本课题选用RTT作为主控芯片的操作系统。由于本课题将制作一个嵌入式系统,所以在下文称RTT描述为嵌入式实时操作系统。





















