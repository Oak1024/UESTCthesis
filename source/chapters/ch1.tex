% !Mode:: "TeX:UTF-8"

\chapter{引言}
\section{USB转接器的背景与意义}
\newacronym[description=通用串行总线]{USB}{USB}{Universal Serial Bus}

键盘是最常见的计算机输入设备,它广泛应用于微型计算机和各种终端设备上,计算机操作者通过键盘向计算机输入各种指令、数据,指挥计算机的工作。计算机的运行情况输出到显示器,操作者可以很方便地利用键盘和显示器与计算机对话,对程序进行修改、编辑,控制和观察计算机的运行。由于USB是键盘的常用接口,因此本课题基于USB接口进行开发。

\gls{USB},是一个外部总线标准,用于规范电脑与外部设备的连接和通讯。是应用在PC领域的接口技术。USB接口支持设备的即插即用和热插拔功能。\gls{USB}是在1994年底由英特尔、康柏、IBM、Microsoft等多家公司联合提出的。

现代键盘普遍使用固件功能的芯片作为USB键盘的主控芯片。这种主控芯片的优点是成本低,稳定性高。缺点则是可扩展性差,无法完全对键盘进行控制。

由于键盘是计算机系统的最主要输入设备,对键盘的完全掌控就显得非常重要。由于缺少对键盘的完全掌控,人们往往需要重复输入很多数据,进行重复劳动,极大地浪费了生产力。因此,本系统通过实现一个USB转接器,为未来键盘的发展提供了一个可行的方向。

\section{国内外的研究现状}
现代键盘和几十年前的键盘相比大部分没有太大区别。少有的几种升级如下:
\begin{enumerate}
\item 蓝牙接口

以前的键盘的接口有AT接口、PS/2接口和最新的USB接口,台式机曾多采用PS/2接口,大多数主板都提供PS/2 键盘接口。而较老的主板常常提供AT 接口也被称为“大口”,已经不常见了。USB作为新型的接口,一些公司迅速推出了USB接口的键盘。由于USB 接口具有热插拔、功能多、速度快等特点,现代电脑普遍使用USB外接键盘。蓝牙接口相比传统键盘,具有无线、可配置等优秀特点,因此被大量用于台式机、平板电脑等不带键盘的电脑设备中。\pic[h]{新型蓝牙键盘}{width=3.5in}{keyboard1}



\item 触摸板

触摸板是一种在平滑的触控板上,利用手指的滑动操作来移动游标的输入装置。当使用者的手指接近触摸板时会使电容量改变,触摸板自身会检测出电容改变量,转换成坐标。其优点在于使用范围较广,全内置、超轻薄笔记本均适用,而且耗电量少,可以提供手写输入功能。随着笔记本的普及,触摸板逐渐被人们所接受,因此出现了带有触摸板的键盘。触摸板通过构造USB复合设备,使一个USB 设备可以拥有触摸板、键盘这两种HID功能。\pic[h]{触摸板键盘}{width=3.5in}{keyboard2}

\item 功能型键盘

功能型键盘的特点是与其他键盘相比多出很多标准之外的键,比如:前进后退键、计算器键、收藏夹键等。功能型键盘的自带功能一般通过安装厂商提供的驱动进行实现。如果没有驱动程序,往往只能实现标准的功能。然而大多数厂商都只提供windows系统下的驱动程序,因此其他平台的用户往往无法使用功能型键盘的全部功能。\pic[h]{功能型键盘}{width=3.5in}{keyboard2}


\end{enumerate}

以上几种键盘都是在原有的键盘的基础上实现了小幅度的改进,增加了一些小功能,在提高用户对键盘的掌控力方面并没有太大的提升,没有让用户从重复劳动中解脱出来,也没用给予用户定制功能的能力。本系统则通过嵌入式技术改变了这一现状——通过实现一个USB转接器增强了用户对键盘的掌控力,真正让用户从计算机输入的重复劳动中解脱出来,极大地扩展了键盘的功能。

\section{理论依据}





















