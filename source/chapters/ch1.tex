% !Mode:: "TeX:UTF-8"

\chapter{引言}
\section{USB转接器的背景与意义}
键盘是最常见的计算机输入设备,它广泛应用于微型计算机和各种终端设备上,计算机操作者通过键盘向计算机输入各种指令、数据,指挥计算机的工作。计算机的运行情况输出到显示器,操作者可以很方便地利用键盘和显示器与计算机对话,对程序进行修改、编辑,控制和观察计算机的运行。

键盘的接口有AT接口、PS/2接口和最新的USB接口,台式机曾多采用PS/2接口,大多数主板都提供PS/2 键盘接口。而较老的主板常常提供AT 接口也被称为“大口”,已经不常见了。USB作为新型的接口,一些公司迅速推出了USB接口的键盘。由于USB 接口具有热插拔、功能多、速度快等特点,现代电脑普遍使用USB外接键盘。

USB,是英文Universal Serial Bus(通用串行总线)的缩写,而其中文简称为“通串线”,是一个外部总线标准,用于规范电脑与外部设备的连接和通讯。是应用在PC领域的接口技术。USB接口支持设备的即插即用和热插拔功能。USB是在1994年底由英特尔、康柏、IBM、Microsoft等多家公司联合提出的。

现代键盘普遍使用固件功能的芯片作为USB键盘的主控芯片。这种主控芯片的优点是成本低,稳定性高。缺点则是可扩展性差,无法完全对键盘进行控制。

要实现对键盘的完全控制,有软件和硬件两种方案。

硬件方案:制作USB转接器,将键盘的USB信息进行处理,再转发给主机,目前没有相关产品上市。


软件方案:AutoHotKey、按键精灵等。

硬件方案和软件相比,具有跨平台(软件和硬件平台)、无安装过程、不占用CPU等特点,缺点则是需要购置费用。

现代常用的通用操作系统有windows、Linux、安卓、MAC、iOS 等。由于系统底层接口差别很大,AutoHotKey与按键精灵这类软件几乎不可能实现跨平台使用,因此硬件方案在某些时候是唯一的选择,因此具有重要意义。

本课题在硬件上将实现类似“按键精灵”的功能,使用精简的AHK语法,给键盘使用者带来跨平台的相同使用体验。
\section{USB转接器的设计内容}
本课题制作了一个基于RTOS的智能USB转接器。其结构图如下:

\pic[h]{转接器示意图}{}{zhuanjieqishiyitu}

如图\ref{zhuanjieqishiyitu}所示,本课题制作了一个USB转接器,同时作为USBHost与USBdev,置于USB-HID设备与USB主机之间,将HID 设备的信息读取后进行处理再发给USB主机。


\subsection{主控芯片选型}
本课题由STM32单片机、OLED显示模块、USB接口等模块组成,各个模块之间通过单片机组成了一个有机的整体。由于本课题使用了USBhost 接口和USBdev 接口,所以需要选择带有USBhost(或OTG)和USBdev接口的单片机。目前不带MMU的单片机没有同时带两个USB 口的,因此需要使用两个单片机进行协作来实现功能。由于作为USBhost的单片机需要实现较多的功能,因此应该选择主频较高的单片机,调查市面上的单片机型号之后,最终确定为STM32F407VG。作为USBdev的单片机只需要实现USBdev功能即可,所以可以选择价格较低的单片机。考虑到开发工具的统一,最终选定的作为USBdev的单片机是STM32F103C8。

\subsection{PCB设计}
由于市面上没有相关产品,本课题需要自行设计PCB并进行焊接。具体的电路设计将在下文中进行介绍。考虑到焊接难度,主控芯片选择的是较容易焊接的QFP 封装。

\subsection{在主控芯片上安装RT-Thread系统}
RT-Thread是新兴的国产实时操作系统(RTOS),在核心板上移植RT-Thread,可以让RT-Thread运行在本课题的硬件上,从而使用RT-Thread带有的系统功能,方便本系统的开发,也能增强系统能力。

\subsection{编写基于RTT系统的相关模块驱动}
由于RT-Thread是嵌入式操作系统,所以操作底层硬件需要编写相应的驱动程序,便于应用层进行调用。

\subsection{编写应用层程序}

应用层程序即功能型程序,就是将本课题需要实现的功能通过编写线程进行实现。

\section{功能简介}
本系统由于是为了应用而开发,所以应用层的工作量比例较大,开发出了多种非常具有实用价值的功能。

\subsection{改键功能}例如:
\verb|通过修改文件系统中的KEY_T键盘映射文件,|可以修改键盘键位映射,语法形式为:“RALT=RCTRL”。这个语句表示右边的ALT 键将映射为右边的CTRL键。各个语句之间并不互相影响,即“RCTRL=RALT RCTRL=RCTRL”之后,相当于没有改键。

\subsection{键盘宏}
键盘宏,顾名思义就是将快捷键定义为其他事件。比如将左边的CTRL+P定义为输出银行卡号码,将右边的ALT+N定义为向下箭头(仿emacs 快捷键)。\verb|语法形式(仿AHK)为:“>^p::{up}”,|意为将右边的ctrl键定义为向上箭头。通过使用键盘宏,可以极大地增强用户对键盘的控制能力,减少关键信息的重复输入(手机号码等)。通过仿emacs定义键位,可以实现全局emacs键位,可以极大地提高用户的输入速度。

\subsection{蓝牙KVM}
KVM:就是Keyboard Video Mouse的缩写,即可以将鼠标键盘视频模拟连接到其他电脑上。本系统只实现了键盘和鼠标的连接。本系统的硬件电路上集成有一个蓝牙模块,可以和其他的相同系统进行蓝牙通信。通过蓝牙,可以使用一套鼠标键盘控制多个硬件,极大地减少了跨平台工作者的操作复杂度。

\subsection{语音识别}
本系统硬件上集成了一个国产语音识别芯片LD3320,可以通过设置拼音进行识别。通过语音识别,可以方便地对电脑进行控制,比如在没有空余手的时候进行语音翻页。

\subsection{鼠标手势识别}
系统应用层通过算法对鼠标的手势进行识别,可以通过模拟键盘实现一些功能(比如最小化窗口,打开计算器等)。

\section{特色与创新}
本系统基于本人需求进行开发,市面上并没有相同的产品出售,因此系统的绝大部分功能都是自主创新。
\subsection{嵌入式操作系统}
本系统基于RT-Thread开发,可扩展性强,可定制性强,且具有多进程处理能力。传统小型嵌入式设备一般直接进行裸机编程,而本系统为了增强系统的能力,基于RT-Thread进行开发,充分利用了RTOS的优势。

\subsection{在线编程}
使用自制脚本分析器,可以进行在线编程,无需仿真器或设备。使得普通用户也能通过修改文件系统内的文件进行复杂的操作,大大增强了系统的可定制程度。

\subsection{跨平台}
通过实现标准USB-HID设备,实现真正的跨平台,Android、MAC、Linux、Windows都通过了测试。


















