% !Mode:: "TeX:UTF-8"

\chapter{功能测试}
\section{系统兼容性测试}
由于本系统的核心优势之一是跨平台,所以笔者测试了本系统在常用的四个带有USB接口的软硬件平台上本系统的功能实现的完整性:
\subsection{ubuntu下测试}
所有功能正常。
\subsection{windows下测试}
windows对USB设备反应速度较慢,所以在极少数情况下会出现USB枚举失败的情况。其他所有功能正常。
\subsection{OS/X下测试}
OS/X系统下鼠标使用不流畅,因此鼠标手势功能测试结果并没有说服力,所以没有进行鼠标手势功能的测试。其他所有功能正常。

\section{功能测试}


鼠标手势功能在定义鼠标手势数量较少的情况下测试正常。如果定义手势较多则由于算法不够完善,容易出现混淆。、

在线编程功能测试正常。不过由于板上SPI接口的FLASH芯片写入速度较慢,所以写入需要的时间较长。如果在写入时进行复位或断电,有可能造成文件系统的损坏。

lua脚本使用上文中的例子进行测试,功能正常。由于Lua使用c语言实现,速度很快,使用时没有延迟的感觉。

\section{硬件外观展示}
\pic[h]{系统内部展示}{width=3in}{pp1}
\pic[h]{系统外观展示}{width=3in}{pp2}








