% !Mode:: "TeX:UTF-8"

\chapter{功能测试}
\section{系统兼容性测试}
由于本系统的核心优势之一是跨平台,所以笔者测试了本系统在常用的四个带有USB接口的软硬件平台上本系统的功能实现的完整性:
\subsection{ubuntu下测试}
所有功能正常。
\subsection{Windows下测试}
windows对USB设备反应速度较慢,所以在极少数情况下会出现USB枚举失败的情况。其他所有功能正常。
\subsection{OS/X下测试}
OS/X系统没有home与end键,所以键盘宏的相应功能无法实现。其他功能正常。



\section{功能测试}
以下功能在Windows7环境下测试。
\begin{enumerate}
\item 改键功能
\threelinetable[htbp]{tabgaijian}{0.3\textwidth}{lcr}{改键功能测试表}
{原始键&目标键&结果\\
}{
CTRL   &ALT   &正常\\
ALT    &SHIFT &正常\\
GUI    &CTRL  &正常\\
SHIFT  &GUI   &正常\\
}{
\item
}

如表\ref{tabgaijian}所示,目前可以实现控制键(CTRL,ALT,GUI,SHIFT)的改键,四个控制键都通过了测试。
\item 键盘宏
\threelinetable[htbp]{tabmacro}{0.35\textwidth}{lcr}{键盘宏功能测试表}
{快捷键&目标效果&结果\\
}{
\verb|SHIFT+1|   &18200258888   &正常\\
\verb|SHIFT+2|    &2011042040022 &正常|\\
\verb|右CTRL+N|    &下一行 &正常|\\
\verb|右CTRL+D|  &del   &正常|\\
}{
\item
}
如表\ref{tabmacro}所示,通过键盘宏进行emacs方向、翻页、换行快捷键模拟,与通过键盘宏输出字符串的功能,都通过了测试。
\item 蓝牙KVM
\threelinetable[htbp]{tabbt}{0.3\textwidth}{lcr}{蓝牙KVM测试表}
{距离(cm)&效果\\
}{
100          &  正常\\
200         &正常\\
300 &有明显延迟\\
400 &无信号\\
}{
\item
}
如表\ref{tabbt}所示,本系统可以在0~2m范围内使用蓝牙KVM功能,足够日常桌面办公使用。
\item 语音识别
\threelinetable[htbp]{tabyuyin}{0.4\textwidth}{lcr}{语音识别功能测试表}
{语音输入&目标效果&成功率\\
}{
向前   &向前翻页   &\verb|10/10|\\
向后    &向后翻页 &\verb|8/10|\\
桌面    &回到桌面  &\verb|6/10|\\
关机  &关机   &\verb|10/10|\\
}{
\item
}
如表\ref{tabyuyin}所示,语音识别功能测试成功率根据具体需要识别的语音而不同。LD3320芯片使用中容易出现接触不良的情况,且MIC 电路焊接容易脱落。

\item Lua脚本

Lua脚本使用上文中的例子进行测试,功能正常。由于Lua使用c语言实现,速度很快,使用时没有延迟的感觉。不过由于板上SPI接口的FLASH芯片写入速度较慢,所以写入需要的时间较长。如果在写入时进行复位或断电,有可能造成文件系统的损坏
\end{enumerate}
\section{硬件外观展示}
\pic[h]{系统内部展示}{width=3in}{pp1}
\pic[h]{系统外观展示}{width=3in}{pp2}
\pic[h]{系统运行时展示}{width=3in}{pp3}








