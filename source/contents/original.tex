% !Mode:: "TeX:UTF-8"

\chapter{USB complete}
\section{author}
Jan Axelson
\section{Timing Constraints and Guarantees}

The allowed delays between the token, data, and handshake packets of a USB
2.0 transaction are very short, intended to allow only for cable delays and
switching times plus a brief time to allow hardware (not firmware) to determine
a response, such as data or a status code, in response to a received packet.

A common mistake in writing firmware is to assume that the firmware should
wait for an interrupt before providing data to send to the host. Instead, before
the host requests the data, the firmware must copy the data to send into the
endpoint’s buffer and arm the endpoint to send the data on receiving an IN
token packet. The interrupt occurs when the transaction completes. After a successful transaction, the interrupt informs the firmware that the endpoint’s
buffer is ready to store data for the next transaction. If the firmware waits for an
interrupt before providing the initial data, the interrupt never happens and data
doesn’t transfer.

A single transaction can carry data bytes up to the maximum packet size the
device specifies for the endpoint. A data packet with fewer than the maximum
packet size’s number of bytes is a short packet. A transfer with multiple transactions can take place over multiple frames or microframes, which don’t have to
be contiguous. For example, in a full-speed bulk transfer of 512 bytes, the maximum number of bytes in a single transaction is 64, so transferring all of the
data requires at least eight transactions, which may occur in one or more
frames.

A data packet that contains a Data PID and error-checking bits but no data
bytes is a zero-length packet(ZLP). A ZLP can indicate the end of a transfer or
successful completion of a control transfer.
\section{Split Transactions}
A USB 2.0 hub communicates with a USB 2.0 host at high speed unless a USB
1.x hub is between the host and hub. When a low- or full-speed device is
attached to a USB 2.0 hub, the hub converts between speeds as needed. But
speed conversion isn’t all a hub does to manage multiple speeds. High speed is
40* faster than full speed and 320* fasterthan low speed. It doesn’t make sense
for the entire bus to wait while a hub exchanges low- or full-speed data with a
device.

The solution is split transactions. A USB 2.0 host uses split transactions when
communicating with a low- or full-speeddevice on a high-speed bus. What
would be a single transaction at low or full speed usually requires two types of
split transactions: one or more start-split transactions to send information to
the device and one or more complete-split transactions to receive information
from the device. The exception is isochronous OUT transactions, which don’t
use complete-split transactions because the device has nothing to send.

Split transactions require more transactions to complete a transfer but make
better use of bus time because they minimize the time spent waiting for a lowor full-speed device to transfer data. The components responsible for performing split transactions are the USB 2.0 host controller and a USB 2.0 hub that
has an upstream connection to a high-speed bus segment and a downstream
connection to a low/full-speed bus segment. The transactions at the device are
identical whether the host is using split transactions or not. At the host, device
drivers and application software don’t have to know or care whether the host is
using split transactions because the protocol is handled at a lower level. Chapter
15 has more about how the host and hubs manage split transactions.
